\documentclass[14pt]{beamer}

\usetheme{Warsaw}

\usepackage{listings}
\usepackage{jlcode}
\lstset{language=julia}

\usepackage{xeCJK}
\setCJKmainfont{HanWangWCL06}


\title{Metaprogramming in Julia}
\author{Iblis Lin}
\institute{}
\date{2018/8/11}


\begin{document}

\frame{\titlepage}

\begin{frame}
\frametitle{Introduction}
  粗淺的分類
  \begin{itemize}
    \item Text-based
      \begin{itemize}
        \item e.g. macro in C
      \end{itemize}
    \item Abstract Syntax Tree Level
      \begin{itemize}
        \item Lisp
        \item Julia
      \end{itemize}
  \end{itemize}
\end{frame}


\begin{frame}{Introduction}
  Metaprogramming 把程式本身視為 data 的一種
\end{frame}


\begin{frame}{Introduction}
  那麼有 data structure 跟 manipulations

  \pause
  在 Julia 中有 \alert{\texttt{Expr}} 這個 type
\end{frame}


\begin{frame}[fragile]{Construct Expressions}

julia 0.7/1.0:
\begin{lstlisting}
  julia> e = Meta.parse("42 + 1")
  :(42 + 1)
\end{lstlisting}

julia 0.6:
\begin{lstlisting}
  julia> e = parse("42 + 1")
  :(42 + 1)
\end{lstlisting}
\end{frame}


\begin{frame}[fragile]{Construct Expressions}
\begin{lstlisting}
  julia> typeof(e)
  Expr
\end{lstlisting}

Fields of \texttt{Expr}:
\begin{itemize}
  \item \texttt{head::Symbol}
  \item \texttt{args::Array\{Any,1\}}
\end{itemize}

\end{frame}


\end{document}
