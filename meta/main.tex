\documentclass[14pt]{beamer}

\usetheme{Warsaw}

\usepackage{listings}
\usepackage{jlcode}
\lstset{language=julia}

\usepackage{xeCJK}
\setCJKmainfont{HanWangWCL06}


\title{Metaprogramming in Julia}
\author{Iblis Lin}
\institute{}
\date{2018/8/11}


\begin{document}

\frame{\titlepage}

\begin{frame}
\frametitle{Introduction}
  粗淺的分類
  \begin{itemize}
    \item Text-based
      \begin{itemize}
        \item e.g. macro in C
      \end{itemize}
    \item Abstract Syntax Tree Level
      \begin{itemize}
        \item Lisp
        \item Julia
      \end{itemize}
  \end{itemize}
\end{frame}


\begin{frame}{Introduction}
  Metaprogramming 把程式本身視為 data 的一種
\end{frame}


\begin{frame}{Introduction}
  那麼有 data structure 跟 manipulations

  \pause
  在 Julia 中有 \alert{\texttt{Expr}} 這個 type
\end{frame}


\begin{frame}[fragile]{Construct Expressions}

julia 0.7/1.0:
\begin{lstlisting}
  julia> e = Meta.parse("42 + 1")
  :(42 + 1)
\end{lstlisting}

julia 0.6:
\begin{lstlisting}
  julia> e = parse("42 + 1")
  :(42 + 1)
\end{lstlisting}
\end{frame}


\begin{frame}[fragile]{Expressions}
\begin{lstlisting}
  julia> typeof(e)
  Expr
\end{lstlisting}

Fields of \texttt{Expr}:
\begin{itemize}
  \item \texttt{head::Symbol}
  \item \texttt{args::Array\{Any,1\}}
\end{itemize}

\end{frame}


\begin{frame}[fragile]{Expressions}
\begin{lstlisting}
julia> e.head
:call

julia> e.args
3-element Array{Any,1}:
   :+
 42
  1
\end{lstlisting}
\end{frame}


\begin{frame}[fragile]{Expressions}
  程式已經用 \texttt{Expr} 來表示了,那麼怎麼執行?

  \pause

\begin{lstlisting}
julia> e
:(42 + 1)

julia> eval(e)
43
\end{lstlisting}
\end{frame}


\begin{frame}[fragile]{Expressions is mutable}

  所以可以各種改。

\begin{lstlisting}
  e.args[1] = :-
  e.args[3] = 50
\end{lstlisting}
\end{frame}


\begin{frame}[fragile]{其他生出 Expression 的方式}

  1. 直接 call constructor

\begin{lstlisting}
  Expr(:call, :+, 2, 3)
\end{lstlisting}

  \pause

  2. Quoting

\begin{lstlisting}
  :(1 + 2)
\end{lstlisting}
\begin{lstlisting}
  quote
    1 + 2
    2 + 3
  end
\end{lstlisting}

\end{frame}


\begin{frame}[fragile]{Printing Expr Instance}
\begin{lstlisting}
  dump(e)
\end{lstlisting}

\begin{lstlisting}
  Meta.show_sexpr(e)
\end{lstlisting}

\end{frame}


\begin{frame}[c]{}
  \centering
  \Large
  The \texttt{Symbol} Type
\end{frame}


\begin{frame}[fragile]{Symbol}
  Like symbol in Lisp or atom in Erlang.

\begin{lstlisting}
  julia> :foo
  :foo

  julia> typeof(:foo)
  Symbol
\end{lstlisting}

  \pause

\begin{lstlisting}
  julia> Symbol("bar-1")
  Symbol("bar-1")
\end{lstlisting}
\end{frame}


\begin{frame}[fragile]{Symbol}
  在 \texttt{Expr} 中的 identifier 會使用 symbol
  \begin{itemize}
    \item variable name
    \item function name
    \item ...
  \end{itemize}

  \pause
\begin{lstlisting}
  julia> e = :(x = 1)
  :(x = 1)

  julia> e.args
  2-element Array{Any,1}:
    :x
   1
\end{lstlisting}
\end{frame}


\begin{frame}[c]{}
  \centering
  \Large
  Interpolation

  \large
  動態的產生 \texttt{Expr} 的手段之一
\end{frame}


\begin{frame}[fragile]{Interpolation}
  用 \texttt{\$} 來 reference 外面的變數

  長得很像 string interpolation

\begin{lstlisting}
  julia> x = 42;

  julia> e = :($x + y)
  :(42 + y)
\end{lstlisting}
\end{frame}


\begin{frame}[fragile]{Splatting Interpolation}
  可以展開整個 array

\begin{lstlisting}
  julia> A
  3-element Array{Symbol,1}:
   :x
   :y
   :z

  julia> e = :(f($(A...)))
  :(f(x, y, z))
\end{lstlisting}
\end{frame}


\begin{frame}[fragile]{Eval and Scope}
  打開 REPL 後,我在哪裡?

  \pause

\begin{lstlisting}
  julia> @__MODULE__
  Main
\end{lstlisting}

  \pause

\begin{lstlisting}
  foo = 1
  e = :(foo += 42)
  eval(e)
\end{lstlisting}

  請問現在 \texttt{foo} 是多少?
\end{frame}


\begin{frame}[fragile]{Eval and Scope}
  \texttt{eval} 的影響範圍是 module 的 global scope

  能夠對 module 的 global scope 有 side effect

  \pause

\begin{lstlisting}
  function f()
    e = :(foo -= 100)
    eval(e)
  end

  foo = 1
  f()
\end{lstlisting}
\end{frame}


\begin{frame}[fragile]{Eval and Scope}
  那麼這個呢?

\begin{lstlisting}
  e = :(bar + 1)
  eval(e)
\end{lstlisting}
\end{frame}


\begin{frame}[fragile]{Example}
  那麼我們現在來實際解決點問題。

  \pause

  現在希望有個 helper function,
  這個 function 能夠對任意的新 \texttt{struct} 建立好看的 \texttt{show} function
\end{frame}


\begin{frame}[fragile]{Example}
\begin{lstlisting}
julia> struct Bar
         a::Int
         b::Bool
         magic::Any
       end

julia> bar = Bar(1, false, "good")
Bar(1, false, "good")

julia> make_show(Bar, :magic, :b)

julia> bar
magic -> good
b -> false
\end{lstlisting}
\end{frame}


\begin{frame}[fragile]{Example}

\begin{lstlisting}[
  basicstyle={\color{jlstring}\ttfamily\scriptsize\selectfont},
  linewidth=350pt,
  basewidth=4pt,
  xleftmargin=-10pt,
  framexleftmargin=0pt,]
function make_show(T::DataType, fields::Symbol...)
  fields = QuoteNode.(fields)
  print_exprs = [ :(println(io, "$($f) -> ", getfield(x, $f))) for f in fields]

  e = :(Base.show(io::IO, x::$T) = $(print_exprs...))
  eval(e)
end
\end{lstlisting}
\end{frame}


\begin{frame}[c]{}
  \centering
  \huge
  Macro
\end{frame}


\begin{frame}[fragile]{Macro}
  Macro 能夠接受參數,而 return 一個 expression,並且\alert{立即執行}這個 expression。

  \pause

\begin{lstlisting}
  julia> macro m()
         :(println("Hello, Macro"))
       end
  @m (macro with 1 method)

  julia> @m()
  Hello, Macro
\end{lstlisting}
\end{frame}


\begin{frame}[fragile]{Macro}

  試試看參數

\begin{lstlisting}
  julia> macro m(name)
         :(println("Hello, ", $name))
       end
  @m (macro with 2 methods)

  julia> @m("Iblis")
  Hello, Iblis
\end{lstlisting}
\end{frame}


\begin{frame}[fragile]{Macro -- Debugging}

  可以用 \texttt{@macroexpand} 這個 macro 來看看你的 macro 回傳了啥

\begin{lstlisting}
  julia> @macroexpand(@m("Iblis"))
  :((Main.println)("Hello, ", "Iblis"))
\end{lstlisting}
\end{frame}


\begin{frame}[fragile]{Macro -- Syntax Sugar}
  在 call 一個 macro 的時候,可以偷懶不寫括號。

\begin{lstlisting}
  julia> @m "Iblis"
  Hello, Iblis
\end{lstlisting}

  而所有參數用空白切開。

  \pause

\begin{lstlisting}
  julia> @macroexpand @m "Iblis"
  :((Main.println)("Hello, ", "Iblis"))
\end{lstlisting}
\end{frame}


\begin{frame}[fragile]{Macro -- Example}

\begin{lstlisting}[
  basicstyle={\color{jlstring}\ttfamily\scriptsize\selectfont},
  linewidth=350pt,
  basewidth=4pt,
  xleftmargin=-10pt,
  framexleftmargin=0pt,]
macro make_show(T::Symbol, fields::Symbol...)
  fields = QuoteNode.(fields)
  print_exprs = [ :(println(io, "$($f) -> ", getfield(x, $f))) for f in fields]
  :(Base.show(io::IO, x::$T) = $(print_exprs...))
end
\end{lstlisting}
\end{frame}


\begin{frame}[fragile]{Macro -- Parse time and Runtime}
\begin{lstlisting}
  macro m(...)
    # parse time
    # ...

    return :(#= excuted at runtime =#)
  end
\end{lstlisting}
\end{frame}


\begin{frame}[c]{}
  \centering
  \huge
  Non-Standard String Literals
\end{frame}


\begin{frame}[fragile]{Non-Standard String Literals}
  這個東西就是透過 macro 完成的

  e.g.

\begin{lstlisting}
  julia> VERSION
  v"1.0.0"

  julia> v"4.2"
  v"4.2.0"

  julia> typeof(v"4.2")
  VersionNumber
\end{lstlisting}
\end{frame}


\begin{frame}[fragile]{Non-Standard String Literals}
\begin{lstlisting}
  julia> using Sockets

  julia> ip"1.1.1.1"
  ip"1.1.1.1
\end{lstlisting}
\end{frame}


\begin{frame}[fragile]{Non-Standard String Literals}
  只要定義 \texttt{@x\_str} 即可。

  \pause

\begin{lstlisting}
  macro foo_str(s)
    :(reverse($s))
  end
\end{lstlisting}

  \pause
\begin{lstlisting}
  julia> foo"abc"
  "cba"
\end{lstlisting}
\end{frame}


\end{document}
